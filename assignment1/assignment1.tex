\documentclass[12pt]{article}
\usepackage[utf8]{inputenc}
\usepackage{hyperref}
\usepackage{amsmath}

\title{
	\Huge{Introduction to Web Science} \\
	\vspace{1em}
	\LARGE{Assignment 1} \\
	\vspace{1em}
	\Large{TANGO}
}

\author {
	Mariya Chkalova \\{\normalsize\href{mailto:mchkalova@uni-koblenz.de}{mchkalova@uni-koblenz.de}} \and
	Arsenii Smyrnov \\{\normalsize\href{mailto:smyrnov@uni-koblenz.de}{smyrnov@uni-koblenz.de}} \and
	Simon Schau\ss \\{\normalsize\href{mailto:sschauss@uni-koblenz.de}{sschauss@uni-koblenz.de}}
}

\date{}

\begin{document}

\maketitle
\pagenumbering{gobble}
\newpage

\pagenumbering{arabic}

\section{Ethernet Frame}

\begin{enumerate}
	\item Source MAC Address: \texttt{00:13:10:e8:dd:52}
	\item Destination MAC Address: \texttt{00:27:10:21:fa:48}
	\item Protocol: Address Resolution Protocol 
	\item The last two blocks of the targets IP Address (\texttt{192.168.2.103}).
\end{enumerate}

\section{Cable Issue}

Let $c$ be the speed of light, $l$ the length of the cable and $t$ the time it takes for the first bit to travel the length $l$. 
As the length of the cables are equal and the networks bandwidth doesn't change the propagation delay, the calculation for both networks are the same.  
Given the speed of light $c = 3 \cdot 10^8 \frac{m}{s}$ and the formula for the propagation delay $t = \frac{l}{c}$, the propagation delay is $t = \frac{20}{3 \cdot 10^8}s \approx 67ns$

\section{Basic Network Tools}

\section{Simple Python Programming}

see \texttt{src/task4.py}

\end{document}
